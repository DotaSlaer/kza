
\documentclass[12pt, a4paper]{extarticle}

\usepackage{polyglossia} % языковой пакет

\usepackage{pdfpages} % пакет для импорта pdf-файлов

\usepackage{tocvsec2} %%%%%%%%%%%%%%%%%%%%%%%%%%%%%%%%%

\usepackage{longtable,booktabs,array}

\usepackage{calc}

\usepackage{ulem}

\usepackage{setspace}

\usepackage[labelsep=period]{caption}

\usepackage{caption}

%\usepackage{slashbox}
\usepackage{diagbox}

\usepackage{graphicx, wrapfig} % пакет для использования графики (чтобы вставлять рисунки, фотографии и пр.)
\renewcommand{\thefigure}{}

\usepackage{floatrow}
\floatsetup[wrapfigure]{capposition=bottom}
\floatsetup[figure]{capposition=bottom,%
capbesideposition={left,bottom},%
capbesidewidth=0.3\textwidth,%
capbesidesep=quad% разделитель между картинкой и подписью
}

% качественные листинги кода
\usepackage{minted}
\usepackage{listings}
\usepackage{lstfiracode}


%\documentclass{article}
\usepackage[12pt]{extsizes}
\usepackage[utf8]{inputenc}
%\usepackage[russian]{babel}
%\usepackage{setspace}
\usepackage{amsfonts}
\usepackage{tikz}
\usepackage{colonequals}


\usepackage[a4paper,top=2cm,bottom=2cm,left=3cm,right=3cm,marginparwidth=1.75cm]{geometry}

% Useful packages
\usepackage{amsmath}
%\usepackage{graphicx}
\usepackage[colorlinks=true, allcolors=blue]{hyperref}

\newcommand*\circled[1]{\tikz[baseline=(char.base)]{
            \node[shape=circle,draw,inner sep=2pt] (char) {#1};}}


\usepackage{amsmath} % поддержка математических символов

\usepackage{url} % поддержка url-ссылок

\usepackage{natbib} % менеджер цитирования natlib.

\bibliographystyle{unsrtnat} % выбираем стиль библиографии отсюда: https://www.overleaf.com/learn/latex/Natbib_bibliography_styles

\setcitestyle{authoryear, open={(},close={)}} % Определяем стиль цитирования. Указываем, чтобы цитирование в тексте вставлялось в формате (Автор, год). 

\usepackage{multirow} % таблицы с объединенными строками

\usepackage{hyperref} % пакет для интеграции гиперссылок

\usepackage{indentfirst} % пакет для отступа абзаца


\usepackage{chngcntr} % пакет подписей и нумерации рисунков

\usepackage{tocloft}

%%%%%%%%%%%%%%%%%%%%%%%%%%%%%%%%%%%%%%%%%%%%%%%%%%%%%%%%%%%%%%

%\usepackage{float}
%\restylefloat{table}

%%%%%%%%%%%%%%%%% Оформление ГОСТА%%%%%%%%%%%%%%%%%

% Все параметры указаны в ГОСТЕ на 2021, а именно:

% Шрифт для курсовой Times New Roman, размер – 14 пт.
\setdefaultlanguage[spelling=modern]{russian}
    \setotherlanguage{english}
    
\setmonofont{Times New Roman}
\setmainfont{Times New Roman} 
\setromanfont{Times New Roman} 
\newfontfamily\cyrillicfont{Times New Roman}




% шрифт для URL-ссылок
\urlstyle{same} 

% Междустрочный интервал должен быть равен 1.5 сантиметра.
\linespread{1.5} % междустрочный интервал


% Каждая новая строка должна начинаться с отступа равного 1.25 сантиметра.
\setlength{\parindent}{1.25cm} % отступ для абзаца


% Текст, который является основным содержанием, должен быть выровнен по ширине по умолчанию включен из-за типа документа в main.tex


%%%%%%%%%%%%%%%%%% Дополнения %%%%%%%%%%%%%%%%%%%%%%%%%%%%%%%%%

% Путь до папки с изображениями
\graphicspath{ {./Images/} }

%Внесение titlepage в учёт счётчика страниц
\makeatletter
\renewenvironment{titlepage} {
	\thispagestyle{empty}
}


% Цвет гиперссылок и цитирования
\usepackage{hyperref} 
 \hypersetup{ 
     colorlinks=true, 
     linkcolor=black, 
     filecolor=blue, 
     citecolor = black,       
     urlcolor=blue, 
     }
    

% Нумерация рисунков
%\counterwithin{figure}{section*}

% Нумерация таблиц
%\counterwithin{table}{section*}

% шрифт для листингов с лигатурами
\setmonofont{FiraCode-Regular.otf}[
	SizeFeatures={Size=10},
	Path = Settings/,
	Contextuals=Alternate
]




% настройка подсветки кода и окружения для листингов
%\usemintedstyle{colorful} % делает подсветку для кода
\newenvironment{code}{\captionsetup{type=listing}}{}


% Посмотреть ещё стили можно тут https://www.overleaf.com/learn/latex/Code_Highlighting_with_minted

\renewcommand{\ULdepth}{1.8pt}

\usepackage[document]{ragged2e}

% для данного титульника слева оступ 3см, справа 1 см, сверху 2см, снизу 0
\usepackage[left=3cm,right=1cm,top=2cm,bottom=0cm]{geometry}

\begin{document}

\fontsize{12pt}{1.08}\selectfont

\begin{center}
\includegraphics[width=0.97917in,height=1.10417in]{Images/mirea.png}

\setlength{\parskip}{6pt}
МИНОБРНАУКИ РОССИИ 

Федеральное государственное бюджетное образовательное учреждение
\setlength{\parskip}{0pt}

высшего образования

\textbf{«МИРЭА -- Российский технологический университет»}

\fontsize{16pt}{1.08}\selectfont
\textbf{РТУ МИРЭА}
\fontsize{12pt}{1.08}\selectfont

\setlength{\parskip}{6pt}
\noindent\rule{\textwidth}{2pt}

\textbf{Институт искусственного интеллекта (ИИИ)}
\setlength{\parskip}{0pt}

\textbf{Кафедра высшей математики (ВМ)}

\bigskip
\textbf{ЗАДАНИЕ}

\textbf{на выполнение курсовой работы}

\end{center}

\justifying

\setlength{\parindent}{0in}
по дисциплине: \uline{Разработка клиентских частей
интернет-ресурсов}

по профилю: \uline{Разработка программных продуктов и проектирование
информационных систем}

направления профессиональной подготовки: \uline{Программная
инженерия (09.03.04)}

\medskip

Студент: \uline{Иванов Иван Иванович}

Группа: \uline{ИКБО-13-19}

Срок представления к защите: \uline{00.12.2020}

Руководитель: \uline{Иванов Иван Иванович, к.т.н., доцент}

\bigskip

\textbf{Тема:} «\uline{Интернет-ресурс на тему «История анимации» с
применением технологий HTML5, CSS3, JavaScript»}

\bigskip

\textbf{Исходные данные:} \uline{используемые технологии: HTML5,
CSS3, JavaScript, текстовый редактор Notepad++/Visual Studio Code/Atom
(на выбор), наличие: интерактивного поведения веб-страниц, межстраничной
навигации, внешнего вида страниц, соответствующего современным
стандартам веб-разработки; инструменты и технологии адаптивной верстки
для полноценного отображения контента на различных браузерах и видах
устройств. Нормативный документ: инструкция по организации и проведению
курсового проектирования СМКО МИРЭА 7.5.1/04.И.05-18.}

\bigskip

\textbf{Перечень вопросов, подлежащих разработке, и обязательного
графического материала:} \uline{1. Провести анализ предметной
области разрабатываемого интернет-ресурса. 2. Обосновать выбор
технологий разработки интернет-ресурса. 3. Создать пять и более
веб-страниц интернет-\\ресурса с использованием технологий HTML5, CSS3 и
JavaScript.\\
4. Организовать межстраничную навигацию. 5. Реализовать слой клиентской
логики веб-страниц с применением технологии JavaScript. 6. Провести
оптимизацию веб-страниц и размещаемого контента для браузеров и
различных видов устройств. 7. Создать презентацию по выполненной
курсовой работе.}

\bigskip

Руководителем произведён инструктаж по технике безопасности,
противопожарной технике и правилам внутреннего распорядка.

\fontsize{12pt}{1.5}\selectfont

\bigskip

Зав. кафедрой ИиППО: \_\_\_\_\_\_\_\_\_\_\_/Р. Г. Болбаков/,
«\_\_\_\_\_»\_\_\_\_\_\_\_\_\_\_\_\_2020 г.

\medskip

Задание на КР выдал: \_\_\_\_\_\_\_\_\_\_\_\_\_\_\_/А.А Иванов/,
«\_\_\_\_\_»\_\_\_\_\_\_\_\_\_\_\_\_2020 г.

\medskip

Задание на КР получил: \_\_\_\_\_\_\_\_\_\_\_/И.И. Иванов/,
«\_\_\_\_\_»\_\_\_\_\_\_\_\_\_\_\_\_2020 г.

\end{document}
